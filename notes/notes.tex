\documentclass{article}
\usepackage[utf8]{inputenc}
\usepackage{amsmath}
\usepackage{graphicx}
\usepackage{amsthm}

\theoremstyle{definition}
\newtheorem{exmp}{Example}[section]
\newtheorem{prop}{Proposition}[section]

\title{Rudin Textbook Notes}
\author{Arjun}

\begin{document}
\maketitle
\begin{center}
 It begins!
 \end{center}
 \newpage
\begin{center}
 Chapter 1: The Real and Complex Number Systems
 \end{center}
 
 \begin{section}
 \noindent The rational numbers are inadequate, both as a field and an unordered set.
 \begin{exmp}
There is no rational number $p$ such that $p^2 = 2$.
\end{exmp}


\begin{proof}
For the sake of contradiction, assume that $p^2 = 2$ has a rational solution, $\frac{a}{b}$ ,where $a$ and $b$ are integers such that $\gcd (a,b) = 1$. Therefore, we can take the square root of both sides as so: 
\[ \sqrt{2} = \frac{a}{b} \]
Through algebraic manipulation, we then get 
\[ 2a^2 = b^2 \]
This means $b$ must be even, and so $b^2$ is divisible by $4$. This means $a$ must also be even. Contradiction, as we assumed that $\gcd (a,b) = 1$.
\end{proof}

\noindent Now, something more interesting: let $A$ be the set of positive rationals $p$ such that $p^2 < 2$, and $B$ be the set of positive rationals $p$ such that $p^2 > 2$.

\begin{prop}
There is NO largest element in $A$.
\end{prop}
\begin{proof}
We have some rational $p$ such that $p^2 < 2$. Now, define a new rational $q$ such that $q = p - \frac{p^2-2}{p-2} = \frac{2(p+1)}{(p+2)}$. Why do we define a rational like so? Well, for one, $q > p$, because $\frac{p^2-2}{p+2}$ is less than zero. Also, $q^2 - 2 = \frac{2(p^2-2)}{(p+2)^2} < 0$. Therefore, for some arbitrary rational $p$ such that $p^2 < 2$, we have found another rational $q$ such that $q > p$ and $q^2 < 2$.
\end{proof}

\begin{prop}
There is NO smallest element in $B$.
\end{prop}
\begin{proof}
Very similar proof. We have some rational $p$ such that $p^2 > 2$. Now, define a new rational $q$ such that $q = p - \frac{p^2-2}{p-2} = \frac{2(p+1)}{(p+2)}$. Why do we define a rational like so? Well, for one, $q < p$, because $\frac{p^2-2}{p+2}$ is greater than zero. Also, $q^2 - 2 = \frac{2(p^2-2}{(p+2)^2} > 0$. Therefore, for some arbitrary rational $p$ such that $p^2 > 2$, we have found another rational $q$ such that $q < p$ and $q^2 > 2$.
\end{proof}

The reason we went through this whole process is to show that even though, say, there's a rational between any two rationals, there are still "gaps" that the rationals have. That's where the real numbers come into play!



\end{section}

 
 
 
\end{document}